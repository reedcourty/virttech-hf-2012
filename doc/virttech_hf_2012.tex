\documentclass[a4paper,10pt,titlepage]{article}

%######################        PREUMBULUM        ###############################

%# Papír méret beállítása:
\usepackage{a4wide}

%###############################################################################
% Lokalizálom:
\usepackage[utf8]{inputenc}
\usepackage[T1]{fontenc}
\usepackage[magyar,english]{babel}
%###############################################################################

%###############################################################################
%#
%#                    DOKUMENTUM METAADATOK:
%#
\newcommand{\szerzo}{Nádudvari György}
\newcommand{\szerzoneptun}{ULQP9P}
\newcommand{\szerzomail}{ulqp9p at gmail dot com}
\newcommand{\cim}{Virtuális gépeket futtató infrastruktúra teljesítményadatainak elemzése vizuális adatelemzés segítségével}
\newcommand{\targy}{vizuális adatelemzés}
\newcommand{\kulcsszavak}{vizuális adatelemzés, vmware, metrikák}
%###############################################################################

%###############################################################################
%#
%#                    NÉHÁNY STÍLUS ÉS FORMÁTUM BEÁLLÍTÁS:
%#
\usepackage{times}
%# Használjunk inkább Arial betűtípust:
%\renewcommand{\rmdefault}{phv} % Arial
%\renewcommand{\sfdefault}{phv} % Arial
%#
%###############################################################################

%###############################################################################
%#
%#                    CSOMAGOK BEHÚZÁSA:
%# Tartalomjegyzék:
\usepackage{tocbibind}
%#
%# Színekhez:
\usepackage[usenames,dvipsnames]{color}
%#
%# Hogy legyen képünk:
\usepackage{graphicx}
%#
%# hyperref használata és beállítása:
\usepackage{hyperref}
\hypersetup{
    unicode=true,
    colorlinks=true,
    linkcolor=RoyalBlue,
    citecolor=RoyalBlue,
    filecolor=RoyalBlue,
    urlcolor=RoyalBlue,
    pdftitle={\cim},
    pdfauthor={\szerzo},
    pdfsubject={\targy},
    pdfkeywords={\kulcsszavak},
}
%#
%# URL-ekhez:
\usepackage{url}
%#
%# Táblázatoknak:
\usepackage{colortbl}
%#
%# Hogy lehessen blokkokat megjegyzéssé tenni:
\usepackage{verbatim}
%#
%# Ettől a táblázatok, ábrak, lábjegyzetek maradnak 1-es sorközzel:
\usepackage{setspace}
%#
%# Saját szín csomag:
\usepackage{texcolors}
%#
%###############################################################################

%###############################################################################
%#
%#                    NÉHÁNY STÍLUS ÉS FORMÁTUM BEÁLLÍTÁS:
%#
\setlength{\parindent}{12pt} % magyar nyelvű dokumentumokban jellemző
\setlength{\parskip}{0pt}    % magyar nyelvű dokumentumokban jellemző

%###############################################################################
%#
%#                    SAJÁT ESZKÖZÖK:
%#
%# Néhány szín definiálása:
\definecolor{todobgszin}{rgb}{0.64,0.78,0.22}
\definecolor{todofrszin}{rgb}{0.00,0.50,0.00}
%#
%# Angol szövegekhez:
\newcommand{\angolul}[1]{\foreignlanguage{english}{#1}}
%#
%# TODO blokk:
\newcommand{\todo}[1]{
    \vfill
    % Csinálunk egy csoportot, hogy az identálás csak erre vonatkozzon:
    \begingroup
        % Beállítjuk, hogy teljes szélességű legyen a dobozunk a bekezdéstől
        % függetlenül:
        \setlength{\parindent}{0cm}
        \fcolorbox{todofrszin}{todobgszin}{
            \parbox{\textwidth}{
                \vskip10pt
                \leftskip10pt
                \rightskip10pt
            
                \emph{TODO: #1}
  
                \vskip10pt
            }
        }
    \endgroup
    \vfill
}
%#
%# Megjegyzés blokk:
\newcommand{\mycomment}[1]{
    \begin{center}
     % Csinálunk egy csoportot, hogy az identálás csak erre vonatkozzon:
    \begingroup
        % Beállítjuk, hogy teljes szélességű legyen a dobozunk a bekezdéstől
        % függetlenül:
        \setlength{\parindent}{0cm}
        \setlength{\textwidth}{12cm}
        \fcolorbox{tc_bone}{tc_eggshell}{
            \parbox{\textwidth}{
                \vskip5pt
                \leftskip5pt
                \rightskip5pt
                
                \small
                
                \textbf{Megjegyzés}
                \newline           
                                
                #1
  
                \vskip5pt
            }
        }
    \endgroup
    \end{center}
}
%#
%# Saját felsorolás stílus:
\newenvironment{sajat_itemize}
{
	\begin{itemize}
	\setlength{\itemsep}{0pt}
}
{
	\end{itemize}
}
%#
%###############################################################################

%###############################################################################
%#
%#                    DOKUMENTUMTÖRZS
%#
\begin{document}

% A magyar nyelv az alapértelmezett:
\selectlanguage{magyar}

%###############################################################################
%#
%#                     CÍMOLDAL:
%#
\begin{titlepage}
    \title{\cim}
    \author{\szerzo \\ (\szerzoneptun) \\ < \szerzomail >}
    \date{\today}
\end{titlepage}
\maketitle
%#
%###############################################################################

% Nem akarom, hogy megjelenjen a tartalomjegyzékben a Tartalomjegyzék:
\section*{Tartalomjegyzék}
\makeatletter
\@starttoc{toc}
\makeatother

\section{Bevezetés}
\subsection{A vizuális adatelemzésről}
\todo{R, ggplot2, Mondrian, előnyök, hátrányok stb.}
\subsection{A vizsgált infrastruktúra}
\todo{Infrastruktúra felépítése, hostok, VM-ek, mérés elvégzésének ideje, metrikák gyűjtésére használt eszközök stb.}
\section{CPU metrikák}

\todo{Forrásként megjelölni: \url{http://www.vmware.com/support/developer/vc-sdk/visdk400pubs/ReferenceGuide/cpu_counters.html}}

\todo{Írni a CPU metrikákról általánosan!}

%###############################################################################
%#
%#                    cpu.swapwait.summation:
%#
\subsection{A cpu.swapwait.summation metrika}

A \texttt{cpu.swapwait.summation} nullától eltérő értéke teljesítménybeli problémát jelent, hiszen azt az időt mutatja, hogy egy swap-in műveletre mennyit kellett várnia egy virtuális gépnek (\ref{tab:cpu.swapwait.summation}.~táblázat). A várakozás oka, hogy hely hiány miatt nem sikerül a virtuális memóriából a memóriába tölteni a működéshez szükséges adatokat.

\Aref{fig:cpu_swapwait_summation}.~ábra a teljes megfigyelt intervallumot ábrázolja. Jól látható, hogy ez idő alatt kétszer fordult elő nagyobb várakozás egy-egy VM-nél. Az egyik augusztus 27-én, a \textit{guest-15} esetében (\ref{fig:cpu_swapwait_summation_082701}.~ábra), a másik szeptember 13-án a \textit{guest-21}-nél(\ref{fig:cpu_swapwait_summation_0913}.~ábra). Amíg a szeptemberi egy viszonylag rövidebb ideig tartott, addig az augusztusit érdemesebb lehet jobban megvizsgálni, hiszen itt egy félórás időszeletről beszélünk. Külön említést érdemel, hogy \aref{fig:cpu_swapwait_summation_082702}.~ábrát jobban megfigyelve a várakozási idő megnövekedése elején nem rendelkezünk minden mérési pontról adattal (a grafikon vonalának lassú az emelkedése).

\begin{figure}[h!]
\centering
\includegraphics[width=1.00\textwidth]{figures/cpu_swapwait_summation-20120826230140-20120924083120.png}
\caption{A cpu.swapwait.summation értékei a monitorozott időintervallumban \label{fig:cpu_swapwait_summation}}
\end{figure}

\begin{figure}[h!]
\centering
\includegraphics[width=1.00\textwidth]{figures/cpu_swapwait_summation-20120913103115-20120913103330.png}
\caption{A cpu.swapwait.summation értéke 2012.09.13. 10:31:15 - 10:33:30 között \label{fig:cpu_swapwait_summation_0913}}
\end{figure}

\begin{figure}[h!]
\centering
\includegraphics[width=1.00\textwidth]{figures/cpu_swapwait_summation-20120827100000-20120827103500.png}
\caption{A cpu.swapwait.summation értéke 2012.08.27. 10:00:00 - 10:35:00 között \label{fig:cpu_swapwait_summation_082701}}
\end{figure}

\begin{figure}[h!]
\centering
\includegraphics[width=1.00\textwidth]{figures/cpu_swapwait_summation-20120827100000-20120827100500.png}
\caption{A cpu.swapwait.summation értéke 2012.08.27. 10:00:00 - 10:05:00 között \label{fig:cpu_swapwait_summation_082702}}
\end{figure}

\begin{table}[h]
	\caption{A cpu.swapwait.summation metrika}
	\centering
	\small
	\begin{tabular}{| p{3.5cm} | p{7.5cm} | p{2cm} |}
		\hline
		\rowcolor{tc_bone} \textbf{Metrika} & \textbf{Leírás} & \textbf{Mértékegység} \\
		\hline
		cpu.swapwait.summation & A 20 milliszekundumos ablakból mennyi időt tölt a VM várakozással (munkavégzés nélkül) amiatt, hogy nem tudja memóriába tölteni a működéséhez szükséges adatokat. & milliszekundum \\ 
		\hline
	\end{tabular}
	\normalsize
	\label{tab:cpu.swapwait.summation}
\end{table}
%#
%###############################################################################

%###############################################################################
%#
%#                    cpu.idle.summation
%#
\subsection{A cpu.idle.summation metrika}

Vizsgálódásunk tárgyai ebben az esetben a \textit{guest-15} és \textit{guest-16} nevű virtuális gépek. \Aref{fig:cpu_idle_summation_g16_1}.~ábrán a \textit{guest-16} VM idle értékei láthatóak a teljes mérési időben. Ahhoz, hogy követeztetéseket tudjunk levonni csökkentenünk kell az időablakot. \Aref{fig:cpu_idle_summation_g16_2}.~ábrával már egyszerűbb dolgunk van. \Aref{fig:cpu_idle_summation_g16_1}.~ábra tömörségének oka, hogy egyfajta periodicitás jellemzi a VM idle állapotban töltött idejét. 

\begin{figure}[h!]
\centering
\includegraphics[width=1.00\textwidth]{figures/cpu_idle_summation-guest-16-20120826230140-20120924083120.png}
\caption{A guest-16 VM cpu.idle.summation értéke a monitorozott időintervallumban \label{fig:cpu_idle_summation_g16_1}}
\end{figure}

\begin{figure}[h!]
\centering
\includegraphics[width=1.00\textwidth]{figures/cpu_idle_summation-guest-16-20120910100000-20120910140000.png}
\caption{A guest-16 VM cpu.idle.summation értéke 2012.09.10. 10:00:00 és 2012.09.10. 14:00:00 között \label{fig:cpu_idle_summation_g16_2}}
\end{figure}

A \textit{guest-15} nevű virtuális gép \texttt{cpu.idle.summation} grafikonja jó példa arra, hogy a vizuális adatelemzés segítségével könnyen kiszúrhatjuk az esetleges mérési hibákat. \Aref{fig:cpu_idle_summation_g15_1}.~ábrán látható, hogy két esetben is (20000 milliszekundumnál) nagyobb értékek találhatóak az adathalmazunkban, mint ahogy annak értelme lenne. \Aref{fig:cpu_idle_summation_g15_2}.~ábra kisebb időablakban mutatja az egyik mérési hiba megjelenését.

\begin{figure}[h!]
\centering
\includegraphics[width=1.00\textwidth]{figures/cpu_idle_summation-guest-15-20120826230140-20120924083120.png}
\caption{A guest-15 VM cpu.idle.summation értéke a monitorozott időintervallumban \label{fig:cpu_idle_summation_g15_1}}
\end{figure}

\begin{figure}[h!]
\centering
\includegraphics[width=1.00\textwidth]{figures/cpu_idle_summation-guest-15-20120919221500-20120919224500.png}
\caption{A guest-15 VM cpu.idle.summation értéke 2012.09.19. 22:15:00 és 2012.09.19. 22:45:00 között \label{fig:cpu_idle_summation_g15_2}}
\end{figure}

\begin{figure}[h!]
\centering
\includegraphics[width=1.00\textwidth]{figures/cpu_idle_summation-max-barchart.png}
\caption{Az egyes VM-ek max. cpu.idle.summation értéke a teljes megfigyelési időintervallumra nézve \label{fig:cpu_idle_summation_max_barchart}}
\end{figure}

Érdemes megnézni az egyes virtuális gépek idle állapotban eltöltött idejének maximumát. Ezt mutatja \aref{fig:cpu_idle_summation_max_barchart}.~ábra.

\mycomment{A már említett mérési hibák nem jelennek meg \aref{fig:cpu_idle_summation_max_barchart}.~ábrán, azok eltávolításra kerültek a diagram adathalmazából.}

\begin{table}[h]
	\caption{A cpu.idle.summation metrika}
	\centering
	\small
	\begin{tabular}{| p{3.5cm} | p{7.5cm} | p{2cm} |}
		\hline
		\rowcolor{tc_bone} \textbf{Metrika} & \textbf{Leírás} & \textbf{Mértékegység} \\
		\hline
		cpu.idle.summation & A 20 milliszekundumos ablakból mennyi időt tölt a VM idle (vagyis futás nélküli) állapotban. & milliszekundum \\ 
		\hline
	\end{tabular}
	\normalsize
	\label{tab:cpu.idle.summation}
\end{table}



\subsection{A cpu.ready.summation metrika}

\todo{Megcsinálni!}

\begin{figure}[h!]
\centering
\includegraphics[width=1.00\textwidth]{figures/cpu_ready_summation-guest-13-20120826230140-20120924083120.png}
\caption{A cpu.ready.summation alakulása a guest-13 azonosítójú VM esetében a teljes megfigyelési időintervallumon \label{fig:cpu_ready_summation_g13}}
\end{figure}

\begin{figure}[h!]
\centering
\includegraphics[width=1.00\textwidth]{figures/cpu_ready_summation-guest-24-20120826230140-20120924083120.png}
\caption{A cpu.ready.summation alakulása a guest-24 azonosítójú VM esetében a teljes megfigyelési időintervallumon \label{fig:cpu_ready_summation_g24_1}}
\end{figure}

\begin{figure}[h!]
\centering
\includegraphics[width=1.00\textwidth]{figures/cpu_ready_summation-guest-24-20120912170000-20120912173000.png}
\caption{A cpu.ready.summation alakulása a guest-24 azonosítójú VM esetében 2012.09.12. 17:00:00 és 2012.09.12. 17:30:00 között \label{fig:cpu_ready_summation_g24_1}}
\end{figure}



\begin{figure}[h!]
\centering
\includegraphics[width=1.00\textwidth]{figures/cpu_ready_summation-max-barchart.png}
\caption{A cpu.ready.summation maximális értéke az egyes VM-ek esetén \label{fig:cpu_ready_summation_max_barchart}}
\end{figure}

\begin{figure}[h!]
\centering
\includegraphics[width=1.00\textwidth]{figures/cpu_ready_summation-histogram.png}
\caption{A VM-ek cpu.ready.summation eloszlása \label{fig:cpu_ready_summation_histogram}}
\end{figure}


\begin{figure}[h!]
\centering
\includegraphics[width=1.00\textwidth]{figures/cpu_ready_summation-histogram-w-b.png}
\caption{A VM-ek cpu.ready.summation eloszlása, az 5 és 10\%-os határokkal \label{fig:cpu_ready_summation_histogram_wb}}
\end{figure}


\begin{table}[h]
	\caption{A cpu.ready.summation metrika}
	\centering
	\small
	\begin{tabular}{| p{3.5cm} | p{7.5cm} | p{2cm} |}
		\hline
		\rowcolor{tc_bone} \textbf{Metrika} & \textbf{Leírás} & \textbf{Mértékegység} \\
		\hline
		cpu.ready.summation & Megmutatja, hogy egy virtuális gép az időablak hány százalékában volt olyan állapotban, hogy ugyan futásra készen állt, de nem került futtatási állapotba a fizikai processzoron. Az érték függ a virtuális gépek számától és azok CPU terheltségétől (loadjától). & milliszekundum \\ 
		\hline
	\end{tabular}
	\normalsize
	\label{tab:cpu.ready.summation}
\end{table}


\subsection{Összefüggés a wait, ready, run metrikák között}

\todo{Megcsinálni!}

\begin{table}[h]
	\caption{A cpu.wait.summation metrika}
	\centering
	\small
	\begin{tabular}{| p{3.5cm} | p{7.5cm} | p{2cm} |}
		\hline
		\rowcolor{tc_bone} \textbf{Metrika} & \textbf{Leírás} & \textbf{Mértékegység} \\
		\hline
		cpu.wait.summation & Értéke megadja, hogy a teljes CPU időhöz viszonyítva mennyi időt töltött a virtuális gép várakozó állapotban. & milliszekundum \\ 
		\hline
	\end{tabular}
	\normalsize
	\label{tab:cpu.wait.summation}
\end{table}

\begin{table}[h]
	\caption{A cpu.run.summation metrika}
	\centering
	\small
	\begin{tabular}{| p{3.5cm} | p{7.5cm} | p{2cm} |}
		\hline
		\rowcolor{tc_bone} \textbf{Metrika} & \textbf{Leírás} & \textbf{Mértékegység} \\
		\hline
		cpu.run.summation & Megadja, hogy a VM az idő hány százalékát töltötte futási állapotban. (Származtatott érték: 100\% = run + ready + wait (+ costop) & milliszekundum \\ 
		\hline
	\end{tabular}
	\normalsize
	\label{tab:cpu.run.summation}
\end{table}

\subsection{Összefüggés az idle, swapwait, system metrikák között}

\todo{Megcsinálni!}

\begin{table}[h]
	\caption{A cpu.system.summation metrika}
	\centering
	\small
	\begin{tabular}{| p{3.5cm} | p{7.5cm} | p{2cm} |}
		\hline
		\rowcolor{tc_bone} \textbf{Metrika} & \textbf{Leírás} & \textbf{Mértékegység} \\
		\hline
		cpu.system.summation & Azoknak az időknek az összege, amelyet a virtuális CPU a virtuális gépben rendszer folyamatok (megszakítás, I/O) kezelésével tölt. Ez nem a vendég operációs rendszer, hanem a kiszolgáló által látott érték. & milliszekundum \\ 
		\hline
	\end{tabular}
	\normalsize
	\label{tab:cpu.system.summation}
\end{table}

\subsection{Korreláció vizsgálata a cpu.usage.average és cpu.usagemhz.average metrikák között}

\section{Memória metrikák}

\todo{Írni a memória metrikákról általánosa?}

\subsection{?}

\todo{Megcsinálni?}

\section{Lemez metrikák}

\todo{Írni a lemezmetrikákról általánosan!}

\subsection{Összefüggés a disk.deviceLatency.average, disk.kernelLatency.average, \\ disk.totallatency.average metrikák között}

\todo{Megcsinálni!}

\subsection{Összefüggés a storagepath, storageadapter, disk latency-k között}

\todo{Megcsinálni!}

\section{Hálózati metrikák}

\todo{Írni a hálózati metrikákról általánosan!}

\subsection{net.transmitted.average}

\todo{Megcsinálni!}

\section{Metrikák összefüggése}

\todo{Megcsinálni!}

\section{Összefoglalás}

\todo{Megcsinálni!}

\end{document}